\documentclass{article}
\usepackage{graphicx} % Required for inserting images
\usepackage{multicol}

\title{Joint Optimization of Trajectory Planning and Task Scheduling in
Heterogeneous Multi-UAV System}
\author{Shubham Thawait 2023csm1015}

\date{}

\begin{document}

\maketitle
\begin{multicols}{2}
\begin{abstract}
    The use of unmanned aerial vehicles (UAV) as a new sensing paradigm is emerging for surveillance and trackingapplications, especially in the infrastructure-less environment. One such application of UAVs is in the construction industry where currently prevalent manual progresstracking results in schedule delays and cost overruns. Inthis paper, we develop a heterogeneous multiUAV framework for progress tracking of large construction sites. The
proposed framework consists of Edge U AV which coordinates the data relay of the visual sensor-equipped Inspection UAVs (I U AV s) to the cloud. Our framework jointly takes into consideration the trajectory optimization of the Edge U AV and the stability of system queues. In particular, we develop a Distance and Access Latency Aware
Trajectory (DLAT) optimization that generates a fair access schedule for I U AV s. In addition, a Lyapunov based
online optimization ensures the system stability of the average queue backlogs for data offloading tasks. Through a
message based mechanism, the coordination between the
set of I U AV s and Edge U AV is ensured without any
dependence on any central entity or message broadcasts.
The performance of our proposed framework with joint
optimization algorithm is validated by extensive simulation results in different parameter settings.
Keyword: Path Planning, Task Scheduling, Data Offloading, Construction Site Monitoring, Unmanned Aerial
Vehicles (UAVs), Lyapunov Optimization
\end{abstract}

\section{Introduction}
The unmanned aerial vehicle (UAV) based solutions are
emerging in various domains such as wireless sensing \cite{mozaffari2019tutorial},
payload delivery , precision agriculture , help and
rescue operations [4], etc. Moreover, with the current
trend of automation, sensing and information exchange
in Industry 4.0, UAV based applications are also finding
their place in the construction industry especially for re-
source tracking and progress monitoring using aerial im-
agery. Such solutions are helpful in infrastructure-less
large construction sites as they provide ease of deploy-
ment, quick access to the ground-truth data and higher
reachability and coverage [5]. Further, the autonomous
or semi-autonomous UAV based solutions could facilitate
progress monitoring, building inspections (for cracks or
other defects), safety inspections (to find any environ-
mental hazards) and many more construction-specific au-
dits automatically. The UAV based visual monitoring of
under-construction projects also allows simultaneous ob-
servability of ground-truth data by different collaborating
entities. Availability of such data and information helps
in timely assessments that could reduce schedule delays,
cost overruns, resource wastage and financial losses which
are not uncommon in construction projects.
A plausible solution to address the aforementioned
challenges could be a Mobile Edge Computing (MEC)
[6] based heterogeneous multi-UAV framework. Such a
framework along with the prior geometric knowledge avail-
able about the construction site as gathered from a Build-
ing Information Model (BIM)[7] could help create an ef-
fective multi-UAV based visual monitoring system for con-
struction sites. As for any constrained environment, the
optimization of computational resources is central to de-
velop a solution. The integration of UAVs and MEC into
a single framework could facilitate that with efficient data
collection/processing from the UAV based dynamic sen-
sors in infrastructure-less environments [8]. In addition,
an MEC based framework can help to perform partial
computation offloading wherein a part of data is processed
by the UAVs while the rest gets offloaded to the cloud.
An MEC based UAV framework is not new and the de-
ployment of the UAVs as base stations or edge servers
is widely studied [9, 10]. These studies reflect on the flexibility in deployment of UAV based edge computing
components. However, there is a problem of buffer over-
flow of UAVs due to the limited on-board processing and
the shared bandwidth to transfer data to the cloud which
leads to instability in the system. In addition, the dy-
namic nature of such systems with varying data traffic
and continuous movement of UAVs makes it difficult to
stabilize or control the system in a deterministic man-
ner. Researchers have used online Lyapunov optimization
[11] to address such system instabilities. Lyapunov opti-
mization considers the stability of the system with time
varying data and optimizes time averages of system utility
and queue backlogs.
In this paper, we address the challenges of deploying
a heterogeneous multi-UAV system for construction site
monitoring by the joint optimization of UAV trajectory
planning and data offloading task scheduling. The pro-
posed framework employs two types of UAVs viz. Inspec-
tion UAVs I U AV s and Mobile Edge UAVs (Edge U AV ).
While the former is deployed as visual sensors to collect
visual data from different locations of the site, the latter
interacts and collects data from I U AV s, and offloads the
same to the cloud. The core objective of the framework
is to minimize the total energy consumption of the sys-
tem while considering the data queue backlogs of I U AV s
and Edge U AV and also jointly optimizing the trajectory
of the Edge U AV in accordance with the trajectories of
I U AV s having minimum access latency and travel dis-
tance. The online resource management such as transmis-
sion power and processor frequency of the Edge U AV is
evolved using Lyapunov optimization (as in [12]).
The rest of the paper is organised as follows: Section
2 presents the proposed heterogeneous multi-UAV frame-
work for construction site monitoring. The overall system
objective is discussed in Sections 3. Sections 4 and 5 dis-
cuss the trajectory optimization and Lyapunov based sys-
tem stability, respectively. The simulation setup has been
presented in Section 6. Section 7 discusses the results
gathered from the experiments while Section 8 concludes
the paper.

\section{Heterogeneous Multi-UAV Frame-
work}
Figure ?? depicts the overall multi-UAV framework with
all its components. The system consists of two hetero-
geneous UAVs i.e. a set of Inspection UAVs I U AV =
{I U AV1, I U AV2, I U AV3, ....., I U AVN } and a Mobile
Edge UAV (Edge U AV ). I U AV s are smaller in size and
are more agile. They collect visual data from a set of
Point of Interests (PoIs) denoted as L = {l1, l2, l3....lk}
across the construction site. As the construction sites are
infrastructure-less environments, there are limited Access
Points (AP) available for connectivity to the cloud. Fur-
ther, the I U AV s possess limited connectivity range that
makes it difficult for them to transfer data to cloud di-
rectly. In addition, the I U AV s move in the 3D Cartesian
coordinate system. The Edge U AV , which is larger in size
and possesses higher computational capabilities, coordi-
nates with the I U AV s to relay the data (after partially
processing the same) to the cloud. Edge U AV always
maintains a constant height and thus its trajectory lies in
an horizontal plane.
The communication between I U AV and Edge U AV
(A2A channel) has limited range and bandwidth. We
have assumed the achievable data transmission rate of
the I U AVi in a given time slot as dof f
i (t). Further, The
height of the Edge U AV is h which is dependent on cov-
erage range r and line of sight (LoS) loss caused due to
environmental effects [13].
The A2A channel power gain (\zeta) from I U AV to
Edge U AV can be given as:

$$ \zeta = g_0*(\frac{dis_0}{dis_1})^\theta $$

 (1)
where g0 is the path loss constant, dis0 is the reference
distance, dist distance between the UAV_s, and \theta is the
path loss exponent.


\subsection{Data collection and offloading}

Each PoI (li) is a tuple (< di, ψi >) where di specifies the
amount of data (images) to be collected and ψi denotes the
coordinates of the site locations in 3D space. The sequence
of PoIs to be visited is provided to I U AV s and same is
also shared with the Edge U AV . During the traversal
along the sequence of PoIs, the limited buffer may make
the I U AV wait at some PoIs along the trajectory until
it offloads the data to the Edge U AV .
The Edge U AV can communicate with one of the
I U AVi in a time slot. The data gathered by each of
the I U AVi in a time slot t is denoted by Ai(t). Qi(t)
represents the queue of the I U AVi and dof f
i (t) denotes
2
the amount of data offloaded to the Edge U AV by the
I U AVi in time-slot t. The recursive equation to update
the Qi(t) is as follows:.
Qi(t + 1) = max{Qi(t) − dof f
i (t), 0} + Ai(t) (2)
The Edge U AV accepts data from the selected I U AVi
in the time-slot t in its queue L(t). The following equation
updates L(t) recursively:
L(t + 1) = max{L(t) − c(t) − dof f
edge(t), 0} + Aedge(t) (3)
where Aedge(t) is the data arrived from the selected
I U AVi in time-slot t, c(t) is the data processed by the
Edge U AV in time-slot t, and dof f
edge(t) is the number of
bits offloaded to the cloud in time-slot t.

\section{System Objective}
n the proposed framework, the offloading of data happens
at two stages - 1) from I U AVi to Edge U AV and 2) from
Edge U AV to the cloud. Our main focus is to achieve the
end-to-end data offloading to the cloud by minimizing the
total energy consumption of the whole system (Esystem)
which is defined as:
Esystem(t) = Etransition
edge (t) + EComm
edge (t)
+
NX
i=1
(EComm
i (t))
! (4)
where Etransition
edge (t) is the transition energy of the
Edge U AV , EComm
edge (t) is a communication energy of the
Edge U AV and EComm
i (t) is the communication energy
of the ithI U AVi.
Further, we discuss the various components of Esystem
along with the expressions to calculate the same.
\subsection{Transition energy of Edge U AV}
The transition energy of Edge U AV refers to the energy
consumed in moving from one location to another. The
transition energy of the Edge U AV is given as:
Etransition
edge = κ||vel(t)||2 (5)
where κ is a constant that depends on the total mass of
the Edge U AV and vel(t) is the velocity of I U AV

\subsection{Communication energy of Edge U AV}
Edge U AV offloads the data to cloud through a wire-
less channel [14]. The communication energy consumed
to transmit the data to the cloud is given as:
Ecomm
edge (t) = (2
dof f
edge(t)
W ∗τ − 1) ∗ N0W
ζ ∗ τ (6)
where the parameters are defined in the Table ??
\subsection{Communication Energy of I U AV}
The energy consumed for offloading the dof f
i (t) data bits
at time slot t from the selected I U AVi to the Edge U AV
using the A2A channel of bandwidth W Hz is given simi-
larly to Equation 6 as:
Ecomm
i (t) = (2 dof f
i (t)
W ∗τ − 1) ∗ N0W
ζ ∗ τ (7)
As the PoIs are predefined and the I U AV s follow a
predetermined path, the energy consumed for the move-
ment of I U AV s are not taken into consideration.
Given the energy of the system, our goal is to find
the optimal parameter values so as to minimize the ex-
pected cumulative energy across the time horizon. The
system policy in every time-slot t can be given by X(t) =
{Fedge(t), pi(t), Pedge(t), Sedge(t)}. Hence, the end-to-end
data offloading policy parameters X(t) aims at minimiz-
ing total expected energy of the system. As the chan-
nel information for the data offloading is not determin-
istic and varies in the environment, the amount of bits
arrived at the Edge U AV depends upon the channel char-
acteristics as well as the current position of the selected
I U AVi. Such time-coupling of variables is responsible for
the stochastic nature of the system. The overall optimiza-
tion model for the stable system performance is given as:
3
P1: min
X(t) lim
T →∞
1
T
TX
t=1
E[Esystem(t)]
s.t.
0 ≤ FM E (t) ≤ F max tϵT (C1)
0 ≤ pi(t) ≤ pi,max i = 1...N tϵT (C2)
0 ≤ Pedge(t) ≤ Pmax tϵT (C3)
dof f
i (t) ≤ Qi(t) i = 1..N tϵT (C4)
c(t) ≤ Fmaxτ
ρedge
tϵT (C5)
dof f
i ≤ W τ log2(1 + (ζ)pi,max(t)
No ∗ W ) i = 1..N tϵT (C6)
dof f
edge(t) ≤ W τ log2(1 + ζPmax(t)
NoW ) tϵT (C7)
lim
T →∞
E[Qi(t)]
T = 0 i = 1..N tϵT (C8)
lim
T →∞
E[L(t)]
T = 0 tϵT (C9)
The constraints C1 and C3 defines the maximum
frequency and maximum transmission power of the
Edge U AV respectively. In addition, C5 defines the max-
imum number of bits processed by Edge U AV . Further-
more, C4 and C6 upper bound the number of transmitted
bits. Similarly, for I U AV , the constraints C2, C4 and C6
bound the number of transmitted bits. The constraints
C8 and C9 establish the rate stability of all system queues
(I U AVi and Edge U AV ). Next we discuss the model to
optimize the trajectory of the Edge U AV with respect to
the trajectories of I U AV s.

\section{Distance and Latency Aware Trajectory}
The flexible and dynamic trajectory planning of
Edge U AV could help in applications within construction
industry where it is hard to reach by terrestrial commu-
nication infrastructure. As already mentioned, the po-
sition of I U AV s changes in every time-slot since they
move through different PoIs to collect data. Hence, the
Edge U AV ’s trajectory needs to be estimated in such a
manner that it can connect and access an I U AVi in a
time-slot before the I U AVi’s queue overflows. Whenever
an I U AVi’s queue gets full, it doesn’t move to its next
designated PoI and sojourns at the same PoI until it is
able to offload its data to the Edge U AV and free up
some queue space. Hence, in order to choose one of the
I U AV s to offload its queue, the Edge U AV would re-
quire the real-time information about the queues of all the
I U AV s in each time-slot. This information is not avail-
able a priori due to the dynamic nature of the system. We
use a message passing based approach for estimating the
queue sizes of the I U AV s in order to make a selection.
Further, the trajectory of the Edge U AV must be opti-
mized so as to consume minimal energy. The trajectory
optimization model of Edge U AV optimizes the trade-off
between transition energy of Edge U AV and access laten-
cies of all I U AVis. In addition, the access latency based
data offloading generates a fair schedule for the I U AV s
to offload data to the Edge U AV . Access latency (Ri(t))
of the ith I U AVi in the time-slot t is the difference be-
tween the time of its last access by the Edge U AV and
the current time-slot.
P2: min
X(t)
TX
t=1
NX
i=1
xi(t) ∗ ||Sedge(t + 1) − Sedge(t)||2
s.t.
||Sedge(t) − Si(t)|| ≤ vmaxτ , i = {1..N } tϵT (C1)
h2
i (t) + ||Sedge(t) − Si(t)||2 ≤gopmax(2 dof f
i
2W τ − 1)−1N −2
o (C2)
Qi(t) ≥ 0, iϵ{1..N }(C3)
NX
i=0
( Ri(t)(1 − xi(t)
(N − 1) ) ≤ Rmax, iϵ{1..N }(C4)
NX
i=0
(xi(t) ∗ Qi(t)) ≥ 1, iϵ{1..N }(C5)
NX
i=0
xi(t) = 1 iϵ1..N (C6)
The first constraint C1 of optimization model P 2 signifies
the distance travelled within a time-slot is limited by the
maximum velocity. The following constraint C2 restricts
that the selected I U AVi should be in the coverage range
of the Edge U AV . Constraint C3 denotes that the queue
of the selected I U AVi shouldn’t be empty while C4 lim-
its the time that has elapsed since the last access of ith
I U AVi should be less than Rmax. The constraint in C5
selects the I U AVi which has data to offload whereas C6
4
is a binary constraint to select only one of the I U AVi in
a time-slot.

\section{Lyapunov Optimization based System Stability}
The model presented in P1 in Section 3 is a stochastic op-
timization problem. The data arrival at the system queues
is random in nature. With the help of online Lyapunov
optimization algorithm, we can solve such stochastic opti-
mization models and jointly stabilize all queues by finding
the optimal X(t) in each time slot [15].
The quadratic Lyapunov function [15] associates a
scalar measure to queues of the system. Further, the sta-
bility of the system is maintained by a guaranteed mean
rate stability of the evolving queues i.e.
lim
T →∞
E[Qi(t)]
T = 0, ∀i ∈ 1, 2, ..N (8)
lim
T →∞
E[L(t)]
T = 0 (9)
Z(v(t)) = 1
2
" NX
i=1
Qi(t)2 + L(t)2
#
(10)
v(t) = [{Qi(t)}N
i=1, L(t)] consists of all backlog queues of
the system at time t and Z(.) is quadratic Lyapunov func-
tion of system queues.
The Lyapunov drift corresponding to above function
can be given as:
△Z(v(t)) = E[(z(v(t + 1)) − z(v(t)))] (11)
The Lyapunov drift plus a penalty function is minimized
to stabilize the queue backlog of the system which is given
as:
△D(t) = △Z(v(t)) + V ∗ E[Esystem(t)] (12)
where V is a positive system constant which controls the
trade-off between Lyapunov drift and the expected energy
of the system. A high value of parameter V signifies more
weight on minimizing energy of the system at the cost of
high queue backlog. Hence, V acts as a trade-off param-
eter between system’s energy and queue backlog.
An upper bound on △Z(v(t)) can be derived as, (for
details see [15, 11])
Z(v(t)) =E[−
NX
i=1
Qi(t) ∗ dof f
i (t)]−
E[L(t) ∗ (c(t) + dof f
edge(t))] + C
(13)
where C is a deterministic constant.
As a result, the upper bound of the drift plus penalty
function becomes
△D(t) ≤ C − E[
NX
i=1
Qi(t)dof f
i (t) − L(t)(c(t) + r(t))]
+V ∗ E[Esystem(t)|v(t)]
(14)
Hence, the original formulation P1 is reduced to P3 which
bounds the system’s drift to keep the system stable as
follows:
P3 min
X(t)E
"
−
NX
i=1
Qi(t) ∗ dof f
i (t)
#
− E[L(t)(c(t)
+dof f
edge(t))] + V E[Esystem(t)]
s.t.
0 ≤ FM E (t) ≤ F max tϵT (C1)
0 ≤ pi(t) ≤ pi,max i = 1...N tϵT (C2)
0 ≤ Pedge(t) ≤ Pmax tϵT (C3)
dof f
i (t) ≤ Qi(t) i = 1..N tϵT (C4)
c(t) ≤ Fmaxτ
ρedge
tϵT (C5)
dof f
i ≤ W τ log2(1 + (ζ)pi,max(t)
No ∗ W )i = 1..N tϵT (C6)
dof f
edge(t) ≤ W τ log2(1 + ζPmax(t)
N0W ) tϵT (C7)
As can be observed, the constraints in P3 is a subset of
the constraints in P1. To further simplify the solution of
the optimization formulation, P3 could be reformulated
as two separate sub-problems provided the positions of
Edge U AV and I U AVi are fixed in a given time slot t.
\subsection{Transmission energy optimization of
I U AV s}First sub-problem deals with the optimization of param-
eters related to the I U AVi. The variables Sedge(t) i.e.
5
position of Edge U AV and the offloaded bits of the se-
lected I U AVi are coupled in particular time interval. The
fixed position of Edge U AV decouples these variables. In
the optimization model P 3.1, the transmission energy
is optimized for a single time-slot given the position of
Edge U AV :
P 3.1 min
pi(t) −
NX
i=1
Qi(t) ∗ dof f
i (t) + V ∗
NX
i=1
pi(t) ∗ τ
s.t.
0 ≤ pi(t) ≤ pi,max i = 1..N tϵT
dof f
i (t) ≤ Qi(t) i = 1..N tϵT
dof f
i ≤ W ∗ τ log2(1 + (ζ) ∗ pi,max(t)
No ∗ W ) i = 1...N tϵT
It can be observed that objective function in P 3.1 is a
convex function. First constraint is linear and the sec-
ond constraint is upper bounded by a concave function.
As a result, the stationary point of the objective func-
tion can be derived as: p∗
i (t) = min{max{ No
ζ ( Qi(t)W
V +
1), 0}, pmax}.
\subsection{Transmission energy optimization of
Edge U AV}
The second sub-problem deals with the optimization of the
Edge U AV parameters for the amount of data offloaded
to the cloud. Further, here we can ignore the processor
frequency parameters and the associated constraints from
the optimization as they do not affect the energy opti-
mization. The updated optimization model is given as:
P 3.2 min
Pedge(t)−L(t)(dof f
edge(t)) + V Pedge(t)τ
s.t.
dof f
edge ≤ L(t)
0 ≤ Pedge(t) ≤ Pmax
r(t) ≤ W τ log2(1 + ζPmax(t)
NoW ) tϵT
The model P 3.2 has a convex optimization objective sub-
ject to convex constraints to solve the optimal transmis-
sion power of the Edge U AV . The stationary point of the
optimization model P 3.2 is Pedge(t) = N0
ζ ( L(t)W τ
V − 1).
The overall solution approach of the proposed hetero-
geneous multi-UAV framework is given in Algorithm 1.
Algorithm 1 Heterogeneous Multi-UAV Framework
Input: Trajectories of all I U AVi and list of PoIs li.
Time, t = 0
while t ≤ T do
1. Estimate the {Qi(t)}N
i=1and {Si(t)}N
i=1
2. Select the ith I U AVi to offload data using P2
3. Compute and offload dof f
i (t) for I U AVi using P 3.1 to
Edge U AV
4. Update Qi(t)
5. Transmit status message to Edge U AV
6. Compute and offload dof f
edge(t) as using P 3.2
7. Update L(t)
8. t=t+1
(a) (b)
Figure 1: (a) 3D Trajectory of UAVs (b) Top View of
Trajectory of I U AV s and Edge U AV for 10 times-
lots with Latency markers

\section{Experimentation}
In this section, we present the simulation setup to val-
idate the efficacy of our proposed Distance and Latency
Aware Trajectory Optimization with Lyapunov based sys-
tem utility. The pre-computed trajectories of each of the
I U AVi are shared with the Edge U AV before the sim-
ulation starts. The simulation parameters are listed in
Table 1.
We have considered a 100m x 100m square region with
PoIs at 2m distance and at heights ranging from 70m
to 80m. There are total 2500 PoIs in the region. We
sample 500 PoIs uniformly at random. Simulation were
performed using three I U AVi and one Edge U AV . All
the I U AVi start from randomly selected PoIs of a ge-
ographic cluster. The sequence of PoIs visited by each
I U AVi is generated using the following steps: 1) Assign
all the I U AVi to randomly selected PoIs of a geographic
cluster. 2) All the I U AVi select the nearest non-visited
PoIs one after the other. This continues till all PoIs are
visited. 3) Before proceeding to the next PoI, an I U AVi
collects all the data (Ai(t)) from that PoI. In this process
of data collection, an I U AVi may remain at the same
PoI across multiple time-slots until all the data (Ai(t)) is
collected. For each PoI, the visual data to be collected is
modelled as the number of bits randomly sampled from a
Gaussian distribution with mean as 800 Kb and variance
as 200 Kb. We conducted separate experiments for low
data and high data scenarios. For low data, the amount
of data at each PoI is two times the output of Gaussian
distribution while for high data, the amount of data to be
collected is 8 times of the Gaussian distribution.
The optimization parameter V ranges from 10 to 1015.
The length of each time slot is 60 seconds. Figure 1a
shows a section of the 3D view of the trajectories followed
by the I U AVi and Edge U AV . Figure 1b shows the top
view of the same with the access latency depicted at each
location point. For better illustration, we have selected a
sequence of 10 time slots to draw the trajectory. As can be
observed in Figure 1b, for I U AV3 the access latency in-
creases from 5 at location (-26,5.7) to 8 (upper threshold)
at location (9,25). Afterwards, the I U AV3 gets accessed
by the Edge U AV resulting in the decrease of access la-
tency to 1 at location (17,19) in the next time slot.
In order to validate the performance of our proposed
framework, we created a baseline approach for both the0
50
100
150
200
250
300
0 1 2 3 4 5 6 7 8 9 10 11 12 13 14
Transmission Power of Edge_UAV
V in Log Scale
DLAT + Lyapunov DLAT + MAX DAT + Lyapunov DAT + MAX
(a) Average Edge UAV Trans-
mission Power consumption0.00E+00
5.00E+07
1.00E+08
1.50E+08
2.00E+08
2.50E+08
3.00E+08
3.50E+08
4.00E+08
0 1 2 3 4 5 6 7 8 9 10 11 12 13 14
Number of Bits
V in Log Scale
DLAT + Lyapunov DLAT + MAX DAT + Lyapunov DAT + MAX
(b) Average Edge UAV Queue
length
Figure 2: Experimental Results on Transmission power
and Queue Length0
10
20
30
40
50
60
0 1 2 3 4 5 6 7 8 9 10 11 12 13 14
Number of Timeslots
V in Log Scale
DLAT + Lyapunov DLAT + MAX DAT + Lyapunov DAT + MAX
Figure 3: Comparison of maximum Access Latency
broad categories of the optimization problems viz.: 1) Tra-
jectory Optimization and 2) Transmission Energy and sys-
tem stability.

\section{Result and Discussions}
n this section, we discuss the comparative performances
of our proposed approach with other baselines.
\subsection{Influence of the trade-off parameter V on
Edge U AV}Figure 2a depicts effect of the increase in the parameter V
with respect to the transmission power of the Edge U AV .
It is evident that DLAT + MAX and DAT + MAX always
consume the maximum energy which makes the average
energy consumption same across different values of V. For
7
DAT + Lyapunov and DLAT + Lyapunov, a drop in the
energy consumption can be observed for V values of 11,
12 and 13. In the Figure 2b, it can be seen that the
average Edge U AV queue length stays low for both DLAT
+ MAX and DAT + MAX at all the values of V. For DAT
+ Lyapunov and DLAT + Lyapunov, we can observe that
the average queue length of Edge U AV starts to increase
around V = 11, 12 and 13. It is to be noted that the
deflection points in the Figure 2a and Figure 2b align with
each other.
In the Figure 2b, the average queue length of
Edge U AV increases with increase in V as the weightage
of the system utility increases. The DLAT and DAT meth-
ods with Lyapunov shows similar performance whereas
the MAX approach is not affected with the change in the
trajectory optimization.
\subsection{Per Time slot analysis of I U AV_i}
As shown in Figure 4 [a], per time slot data offloading
schedule of I U AVi based on the trajectory of Edge U AV
incurs higher access latency for DAT based approaches.
Besides this, Figure 4 [b] shows that the queue of I U AVi
is higher for the DAT combinations throughout. It is in-
teresting to note from these results that the queue uti-
lization in DLAT combinations is well spread out keeping
the energy consumption less for optimal V. However, for
DAT combinations the queue utilization is more bursty
in nature and so is the energy consumption which even
touches the MAX baseline for some time-slots as shown
in Figure 4 [c]. This behavior can be explained by the
fact that in the DAT based approaches, the Edge U AV
selects the nearest I U AV which may not have sufficient
data to offload at that time instant. On the other hand,
the DLAT based approaches select the I U AV s optimally
considering the distance as well as the data availability in
the I U AV queues.
Hence, it can seen that our proposed DLAT + Lya-
punov shows consistently better performances with opti-
mally balanced trade-off between the trajectory optimiza-
tion with low access latencies and minimal transmission
power consumption of the system.

\section{Conclusion}
AV based applications for progress monitoring and re-
source tracking are emerging in construction industry.Constructions projects have minimal infrastructure for
capture and offloading of ground-truth data. This paper
presents a heterogeneous multi-UAV based framework for
end-to-end data collection and offloading using a distance
and latency aware trajectory optimization. The Lyapunov
optimization approach is used to ensure the stability of
the system in terms of expected system queue backlogs by
breaking the system optimization problem into two sub-
problems. The simulation results show that the access
latency of our proposed (DLAT + Lyapunov) approach
performs better than other baseline approaches. More-
over, the analysis of system parameter V has shown a
trade-off between the queue stability and the system util-
ity.
\bibliographystyle{plain}
\bibliography{1}

\end{multicols}
\end{document}

https://www.overleaf.com/project/64cc90055a4a21d0a50c9b34